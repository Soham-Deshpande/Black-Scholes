\documentclass[12pt]{article}
\usepackage{lingmacros}
\usepackage{tree-dvips}
\begin{document}

\title{Options Pricing Research }
\author{Soham Deshpande}
\maketitle
\clearpage
\tableofcontents
\clearpage

\section{Options}
Options are a versatile financial product that are based on the value of the 
underlying security. These contracts offers the buyer an opportunity to buy or
sell, but unlike a future, the contact holder is not required to buy or sell the 
commodity.Through paying a premium, a buyer can gain the rights granted by the
contract.
\\
There are two types of options: call and put options. A call options is the
right to buy
or take a long position in a given asset. A put options is the right to sell or
take a short position in a given asset. The asset to be bought or solder under
the terms of the options is the underlying asset. The price at which the
underlying will be delivered is called the Strike Price. The date after which the
option may no longer be exercised is the expiration date.
\\
The contact specifications contain the following: Underlying asset, expriation
date, exercise price and type. 
 
\section{Binomial Pricing Model}

\end{document}
