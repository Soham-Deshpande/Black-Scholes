\documentclass{article}


%defining all packages
\usepackage[utf8]{inputenc}
\usepackage{amsmath}
\usepackage{graphicx}
\usepackage{amsfonts}
\usepackage{algorithm}
\usepackage{adjustbox}
\usepackage{listings}
\usepackage{color}
\usepackage{algpseudocode}
%end of packages


%title page
\title{Investigation into Black Scholes Model}

\author{Soham Deshpande}
\date{October 2022}

\begin{document}

\maketitle
\clearpage
\tableofcontents
\clearpage
%end of title page

%Explanation of Options
\section{Introduction to Options}
The stock market offers many ways to trade between the parties, one of which
is the concept of options. Options can be defined as contracts that allow an
investor to buy or sell an underlying security at a predetermined price, called
the strike price, on a predetermined day. Unlike futures, the holder is not
required to buy or sell the asset if they decide against it.
Each options contract will have a specific expiration date by which the holder
must exerice their option. There are 2 types of options: Call and Puts. \\

\subsection{Calls}

\subsection{Puts}

\clearpage
\section{Ito's Lemma}


\clearpage
\section{Heat Equation}


\clearpage
\section{Black Scholes Equation}


\clearpage
\section{Implementation}


\clearpage
\subsection{Python}


\clearpage
\subsection{C++}









\end{document}
